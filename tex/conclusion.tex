\section{Conclusion}

In this work, we discussed the theoretical and practical limits of compression algorithms for center-based clustering. We proposed the first nearly-linear time
coreset algorithm for $k$-median and $k$-means. Moreover, the algorithm can be parameterized to achieve an asymptotically optimal coreset size. Subsequently, we
conducted a thorough experimental analysis comparing this algorithm with fast sampling heuristics. In doing so, we find that although the Fast-Coreset algorithm
achieves the best compression guarantees among its competitors, naive uniform sampling is already a sufficient compression for downstream clustering tasks in
well-behaved datasets. Furthermore, we find that intermediate heuristics interpolating between uniform sampling and coresets play an important role in
balancing efficiency and accuracy. 

Although this closes the door on the highly-studied problem of optimally small and fast coresets for $k$-median and $k$-means, open questions of wider scope
still remain. For example, under what conditions does sensitivity sampling guarantee accurate compression in linear-time or of optimal space and can these
generalizations be formalized? Furthermore, sensitivity sampling is incompatible with paradigms such as fair-clustering and it is unclear whether one can
expect that a linear-time method can optimally compress a dataset while adhering to the fairness constraints.
