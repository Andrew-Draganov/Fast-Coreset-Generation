\section{Introduction}

The modern data analyst has no shortage of clustering algorithms to choose from but, given the ever-increasing size of relevant datasets, many are often too
slow to be practically useful.  Indeed, Llyod's method \cite{Lloyd82}, arguably the most popular clustering algorithm, runs for multiple iterations until
convergence with every iteration requiring $O(ndk)$ time, where $n$ is the number of points, $d$ is the number of features and $k$ is the number of clusters.
Naturally, this is prohibitively slow whenever the data is very large. Examples like these have prompted the rise of big data algorithms that provide both
theoretical and practical improvements for standard data-science techniques. The perspectives of theoretical soundness and practical efficacy are, however,
often at odds with one another. On the one hand, theoretical guarantees provide assurance that the algorithm will obtain valid solutions without unlucky inputs
affecting their running time or accuracy. On the other hand, it is sometimes difficult to convince oneself to implement the theoretically optimal algorithm when
there are cruder methods that are faster to get running and perform well in practice.

Since datasets can be large in the number of points $n$ and/or the number of features $d$, big-data methods must mitigate the effects of both.  With respect to
the feature space, the question is effectively closed as random projections are fast (running in effectively linear time), practical to implement
\cite{MakarychevMR19}, and provide tight guarantees on the embedding's size and quality. The outlook is less clear when reducing the number of points
$n$, and there are two separate paradigms that each provide distinct advantages.  On the one hand, we have uniform sampling, which runs in sublinear time but
may miss important subsets of the data and therefore can only guarantee accuracy under certain assumptions on the data \cite{HJJ23}.  On the other hand, the
most accurate sampling strategies are those that provide the \emph{strong coreset} guarantee, wherein the cost of any solution on the compressed data is within
an $\varepsilon$-factor of that solution's cost on the original dataset \cite{CSS21}.  Surprisingly, many coreset constructions can remove the dependency on
$n$, significantly accelerating downstream tasks.

We study both paradigms with respect to a classic problem -- what are the limits and possibilities of compression for the $k$-means and $k$-median objectives?
Whereas uniform sampling provides optimal speed but no worst-case accuracy guarantee, all coreset constructions have a running time of at least
$\tilde{O}(nd+nk)$ when yielding tight bounds on the minimum number of samples required for accurate compression. 

It is easy to show that any algorithm that achieves a compression guarantee must read the entire dataset\footnote{A simple example is to consider a data matrix
in $\mathbb{R}^{n \times d}$ with one single large entry and the remaining entries being very small. The algorithm must find this entry to yield a good
2-clustering.}. Thus a clear open question is what guarantees are achievable in linear or nearly-linear time. Indeed, currently available fast sampLing
algorithms for clustering \cite{lightweight_coresets} \cite{kmeans_sublinear_bachem16} cannot achieve strong coreset guarantees.  Recently, \cite{DSWY22} proposed
a sensitivity-based method for strong coresets that runs in time $\tilde{O}(nd + nk)$ and conjectured this to be optimal for $k$-median and $k$-means.  Since
the issue of determining a coreset size of optimal size has recently been closed \cite{CSS21,CLSSS22,huangLB}, this is arguably the main open problem in coreset
research for center-based clustering. We resolve this by showing that there exists an easy-to-implement algorithm that constructs coresets in $\tilde{O}(nd)$
time -- only logarithmic factors away from the time it takes to read in the dataset.

Nonetheless, this does not fully illuminate the landscape among sampling algorithms for clustering in practice. Although our algorithm achieves both an optimal
runtime and an optimal compression, it is certainly possible that other, cruder methods may be just as viable for all practical purposes.  Thus, there is
a natural spectrum among the linear-time algorithms -- obtaining more information about the dataset allows for a better sampling strategy at the expense of
sublinear factors in speed. We state this formally in the following question: When are optimal $k$-means and $k$-median coresets necessary and what is the
practical tradeoff between coreset speed and accuracy?

To this end, we identify three sampling algorithms that are faster than our proposed method and perform a thorough comparison across all of them. Through this,
we verify that, while faster methods are sufficiently accurate on many real-world datasets, there exist data distributions that cause catastrophic failure for
each of them. Indeed, these cases can only be avoided with a strong-coreset method. Thus, while many practical settings do not require the full coreset
guarantee, one cannot cut corners if one wants to be confident in their compression. We verify that this extends to the streaming paradigm and applies to
downstream clustering approaches.

In summary, our contributions are as follows:
\begin{itemize}

    \item We show that one can obtain strong coresets for $k$-means and $k$-median in $\tilde{O}(nd)$ time. This resolves a conjecture on the necessary runtime
        for $k$-means coresets \cite{DSWY22} and is theoretically optimal up to log-factors.

    \item Through a comprehensive analysis across datasets, tasks, and streaming/non-streaming paradigms, we verify that there is a necessary tradeoff between
        speed and accuracy among the linear- and sublinear-time sampling methods. This gives the clustering practitioner a blueprint on when to use each
        compression algorithm for effective clustering results in the fastest possible time.

\end{itemize}
