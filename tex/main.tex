%%%%%%%% ICML 2022 EXAMPLE LATEX SUBMISSION FILE %%%%%%%%%%%%%%%%%

\documentclass[nohyperref]{article}

% Recommended, but optional, packages for figures and better typesetting:
\usepackage{microtype}
\usepackage{graphicx}
\usepackage{subfigure}
\usepackage{booktabs} % for professional tables

% hyperref makes hyperlinks in the resulting PDF.
% If your build breaks (sometimes temporarily if a hyperlink spans a page)
% please comment out the following usepackage line and replace
% \usepackage{icml2022} with \usepackage[nohyperref]{icml2022} above.
\usepackage{hyperref}


% Attempt to make hyperref and algorithmic work together better:
\newcommand{\theHalgorithm}{\arabic{algorithm}}

% Use the following line for the initial blind version submitted for review:
\usepackage{icml2022}

% If accepted, instead use the following line for the camera-ready submission:
% \usepackage[accepted]{icml2022}

% For theorems and such
\usepackage{amsmath}
\usepackage{amssymb}
\usepackage{mathtools}
\usepackage{amsthm}

% if you use cleveref..
\usepackage[capitalize,noabbrev]{cleveref}

%%%%%%%%%%%%%%%%%%%%%%%%%%%%%%%%
% THEOREMS
%%%%%%%%%%%%%%%%%%%%%%%%%%%%%%%%
\theoremstyle{plain}
\newtheorem{theorem}{Theorem}[section]
\newtheorem{proposition}[theorem]{Proposition}
\newtheorem{lemma}[theorem]{Lemma}
\newtheorem{corollary}[theorem]{Corollary}
\theoremstyle{definition}
\newtheorem{definition}[theorem]{Definition}
\newtheorem{assumption}[theorem]{Assumption}
\theoremstyle{remark}
\newtheorem{remark}[theorem]{Remark}

% Todonotes is useful during development; simply uncomment the next line
%    and comment out the line below the next line to turn off comments
%\usepackage[disable,textsize=tiny]{todonotes}
\usepackage[textsize=tiny]{todonotes}


% The \icmltitle you define below is probably too long as a header.
% Therefore, a short form for the running title is supplied here:
\icmltitlerunning{Linear-Time Construction of Coreset}


\DeclareMathOperator{\polylog}{polylog}
\DeclareMathOperator{\cost}{cost}
\DeclareMathOperator{\dist}{dist}


\newcommand{\R}{\mathbb{R}}
\newcommand{\E}{\mathbb{E}}
\newcommand{\coreset}{\Omega}
\newcommand{\calC}{\mathcal C}
\newcommand{\calA}{\mathcal A}
\newcommand{\opt}{\text{OPT}}
\newcommand{\eps}{\varepsilon}

\newcommand{\fkmeans}{\textsc{Fast-kmeans++}}

\newcommand{\lpar}{\left(}
\newcommand{\rpar}{\right)}
\newcommand{\lbra}{\left\{}
\newcommand{\rbra}{\right\}}
\newcommand{\lnor}{\left\|}
\newcommand{\rnor}{\right\|}


\newcounter{sideremark}
\newcommand{\marrow}{\stepcounter{sideremark}\marginpar{$\boldsymbol{
\longleftarrow\scriptstyle\arabic{sideremark}}$}}

 \newcommand{\david}[1]{
 %  \ifdraft{
    \textsf{{\color{blue} *** (David) \marrow #1 ***}}
 %  }
 %  \fi
 }
 
 
%for having enumerates with (i) (ii) (iii): 
 \renewcommand{\labelenumi}{\theenumi}
 
\begin{document}

\twocolumn[
\icmltitle{Linear-Time Construction of Coreset for Clustering}

% It is OKAY to include author information, even for blind
% submissions: the style file will automatically remove it for you
% unless you've provided the [accepted] option to the icml2022
% package.

% List of affiliations: The first argument should be a (short)
% identifier you will use later to specify author affiliations
% Academic affiliations should list Department, University, City, Region, Country
% Industry affiliations should list Company, City, Region, Country

% You can specify symbols, otherwise they are numbered in order.
% Ideally, you should not use this facility. Affiliations will be numbered
% in order of appearance and this is the preferred way.
\icmlsetsymbol{equal}{*}

\begin{icmlauthorlist}
\icmlauthor{Vincent Cohen-Addad}{equal,google}
\icmlauthor{Andrew Draganov}{equal,aarhus}
\icmlauthor{David Saulpic}{equal, univie}
\icmlauthor{Chris Schwiegelshohn}{equal, aarhus}
\end{icmlauthorlist}

\icmlaffiliation{google}{Google Research, Franche}
\icmlaffiliation{aarhus}{Aarhus University, Denmark}
\icmlaffiliation{univie}{University of Vienna, Austria}
%
\icmlcorrespondingauthor{David Saulpic}{david.saulpic@lip6.fr}
%\icmlcorrespondingauthor{Firstname2 Lastname2}{first2.last2@www.uk}

% You may provide any keywords that you
% find helpful for describing your paper; these are used to populate
% the "keywords" metadata in the PDF but will not be shown in the document
\icmlkeywords{coreset, clustering, k-means}

\vskip 0.3in
]

% this must go after the closing bracket ] following \twocolumn[ ...

% This command actually creates the footnote in the first column
% listing the affiliations and the copyright notice.
% The command takes one argument, which is text to display at the start of the footnote.
% The \icmlEqualContribution command is standard text for equal contribution.
% Remove it (just {}) if you do not need this facility.

%\printAffiliationsAndNotice{}  % leave blank if no need to mention equal contribution
\printAffiliationsAndNotice{\icmlEqualContribution} % otherwise use the standard text.

\begin{abstract}
We study the ways of compressing a dataset of points in $\R^d$ while preserving some guarantee. For this problem, show how to implement the so called sensitivity sampling in near-linear time $\tilde O (nd)$. 
This sampling distribution ensures that the sample forms a coreset for $k$-median and $k$-means, namely, it preserves the cost with respect to any $k$-median or $k$-means solution.
We implement this algorithm and illustrates its efficiency on real-life datasets. 
In particular, it is faster than any coreset algorithm with provable guarantees, and it shows way better guarantees than uniform sampling.
%This document provides a basic paper template and submission guidelines.
%Abstracts must be a single paragraph, ideally between 4--6 sentences long.
%Gross violations will trigger corrections at the camera-ready phase.
\end{abstract}

\section{Introduction}

The modern data analyst has no shortage of clustering algorithms to choose from but, given the ever-increasing size of relevant datasets, many are often
too slow to be practically useful. This has prompted the rise of big data algorithms which provide both theoretical guarantees and
practical improvements for standard data-science techniques on otherwise insurmountable datasets. The perspectives
of theoretical soundness and practical efficacy are, however, slightly at odds with one another. On the one hand, theoretical guarantees provide assurance that
the algorithm will obtain valid solutions without unlucky inputs affecting their runtime or accuracy. On the other hand, it is sometimes difficult to convince
oneself to implement the theoretically optimal algorithm when there are cruder methods that are faster to get running and perform well in practice.

Since datasets can be large in the number of points $n$ and/or the number of features $d$, big-data methods must mitigate the effects of both.
With respect to the feature space, the question is effectively closed as random projections are fast (running in effectively linear time), practical to
implement, and provide tight guarantees on the embedding's size and quality. The outlook is less clear when reducing the number of points $n$, and there are
two separate paradigms that each provide distinct advantages.  On the one hand we have uniform sampling, which runs in sublinear time but may miss important subsets of
the data and therefore cannot guarantee accuracy.  On the other hand the most accurate sampling strategies are those that provide the \emph{strong coreset}
guarantee, wherein the cost of any solution on the compressed data is within an $\varepsilon$-factor of that solution's cost on the original dataset.
Surprisingly, many coreset constructions can remove the dependency on $n$, significantly accelerating downstream tasks.

We study both paradigms with respect to a classic problem -- what are the limits and possibilities of compression for the $k$-means and $k$-median objectives?
Whereas uniform sampling provides optimal speed but no accuracy guarantee, coreset constructions have
unclear runtime bounds despite their tight bounds on the minimum number of samples required for accurate compression. Although there is little to study with respect to random
sampling, the lack of clarity regarding linear-time coreset constructions throws into question their ability to facilitate clustering on large datasets. Indeed,
available linear-time coreset via sensitivity sampling obtain an additive error \cite{BachemL018} while other fast sampling strategies
\cite{kmeans_sublinear_bachem16} have weaker guarantees. Recently, \cite{DSWY22} proposed a sensitivity-based method for strong coresets that runs in time
$\tilde{O}(nd + nk)$ and conjectured this to be optimal for $k$-median and $k$-means.  Since the issue of determining an optimal coreset size has recently
been resolved \cite{CSS21,CLSSS22,HLW23}, this is arguably the main open problem in coreset research for center-based clustering. We resolve this by showing that
there exists an easy-to-implement algorithm that constructs coresets in $\tilde{O}(nd)$ time -- only logarithmic factors away from the time it takes to read in
the dataset.

Nonetheless, this does not fully illuminate the landscape between the sampling paradigms for clustering algorithms when applying them in practice. Although our algorithm achieves an optimal
running time, it is certainly possible that other, cruder methods may be just as viable on most real world data sets despite having no worst case guarantees.
Thus, there is a natural spectrum within linear-time algorithms where obtaining more information about the dataset allows for a better
sampling strategy at the expense of sublinear factors in speed. We state this formally in the following question: When are optimal $k$-means and $k$-median
coresets necessary and what is the practical tradeoff between coreset speed and accuracy? To this end, we show that while many practical settings do not require
the full coreset guarantee, one cannot cut corners if one wants to be confident in their compression. We also
show that this extends to the streaming data paradigm and applies to downstream clustering approaches.

\section{Coreset Algorithms}
\label{sec:theory}

In this section, we first combine two existing results to produce a strong coreset in time $\tilde{O}(nd \log \Delta)$, where $\Delta$ is the aspect-ratio of
the input.  We show afterwards how to reduce the dependency in $\Delta$ to $\log \log \Delta$, giving the desired nearly-linear runtime.

Our method is based on the following observations about the group sampling \cite{stoc21} and sensitivity sampling \cite{FeldmanL11} coreset construction
algorithms. Both start by computing a solution $\calC$. When $\calC$ is a $c$-approximation, they compute a $c \eps$-coreset of size $\tilde O\lpar
k \eps^{-z-2}\rpar$ and $\tilde O\lpar k \eps^{-2z -2}\rpar$, respectively. Hence, by rescaling, they provide $\eps$-coreset with size $\tilde O\lpar
k (\eps/c)^{-z-2}\rpar$ and $\tilde O\lpar k (\eps/c)^{-2z-2}\rpar$. 

This leads to the following fact:
\begin{fact}\label{fact:logApprox}
Let $\calC$ be an $O\lpar \log^{O(1)} k\rpar$ approximation to $k$-median or $k$-means.
Then, group sampling using solution $\calC$ computes a coreset of size $\tilde O\lpar
k \eps^{-z-2}\rpar$, and sensitivity sampling one of size $\tilde O\lpar k \eps^{-2z-2}\rpar$. 
Both runs in time $\tilde O(nd)$, provided $\calC$.
\end{fact}

To turn \cref{fact:logApprox} into an algorithm, we use the \fkmeans approximation algorithm from \cite{cohen2020fast}, which has the two following key properties: 
\begin{itemize}
\item \fkmeans runs in $\tilde O\lpar n d \log \Delta\rpar$ time (Corollary 4.3 in \cite{cohen2020fast}), and
\item \fkmeans computes an assignment from input point to centers that is a $O\lpar d^z \log k\rpar$ approximation to $k$-median ($z=1$) and
$k$-means ($z=2$) (Lemma 3.1 in \cite{cohen2020fast} for $z=2$, the discussion above for $z=1$). Applying dimension reduction techniques \cite{MakarychevMR19}, the dimension $d$ may be replaced by a $\log k$. This results in a $O\lpar\log^{z+1} k\rpar$ approximation.
\end{itemize}

The second property is crucial for us: the algorithm does not only compute centers, but also assignments in $\tilde{O}(nd\log \Delta)$ time.  We use it, in
combination with sensitivity sampling, as described in \cref{alg:main}.  This algorithm computes an $\eps$-coreset in time $\tilde O(nd \log \Delta)$: we prove
formally the statement in \cref{app:theory} Thus, it is easy to combine existing results to obtain an $\eps$-coreset without an $\tilde{O}(nk)$ time-dependency.
However, our method thus far has only replaced the $\tilde{O}(nd + nk)$ with an $\tilde{O}(nd \log \Delta)$. Indeed, the spirit of the issue remains -- this is
not log-linear in the input dimension.  We verify this in Table \ref{tbl:logdelta}, where the runtime grows linearly with the ratio of the maximum and minimum
distance.

\begin{algorithm}[tb]
   \caption{Fast-Coreset($P, k, \eps, S$)}
   \label{alg:main}
\begin{algorithmic}[1]
   \State {\bfseries Input:} data $P$, number of clusters $k$, precision $\eps$ and target size $S$
   \State Use Johnson-Lindenstrauss embedding to compute the embedding $\tilde P$ of $P$ into $\tilde d = O(\log k)$ dimensions
   \State Compute an approximate solution $\tilde \calC = \lbra \tilde c_1, ..., \tilde c_k\rbra $ for $\tilde P$, and an assignment $\tilde \sigma : \tilde P \rightarrow \tilde \calC$ using \fkmeans.	
   \State Let $\calC_i = \tilde \sigma^{-1}(c_i)$. Compute the $1$-median (or $1$-mean) $c_i$ of each $\calC_i$ in $\R^d$.%, and define $\sigma(p) := c_i$ for all $p \in \calC_i$.
   \State For each point $p \in \calC_i$, define
   $s(p) = \frac{\dist^z(p, c_i)}{\cost(\calC_i, c_i)}+ \frac{1}{|\calC_i|}$.
   \State Compute a set $\coreset$ of $S$ points randomly sampled from $P$ proportionate to $s$.
   \State For each $\calC_i$, define $|\hat \calC_i|$ the estimated weight of $\calC_i$ by $\coreset$, namely $|\hat \calC_i| := \sum_{p \in \calC_i \cap \coreset} \frac{\sum_{p' \in P}s(p')}{s(p)S}$.
   \State {\bfseries Output:} the coreset $\coreset$, with weights $w(p) = \frac{\sum_{p' \in P}s(p')}{s(p)S} \lpar (1+\eps)|\calC_i| - |\hat \calC_i|\rpar$
\end{algorithmic}
\end{algorithm}

\begin{table}[htbp]
    \centering
    \begin{tabular}{lrrrr}
        \hline
        Value of $r$ & 20 & 30 & 40 & 50 \\
        Runtime (s) & 13.5 & 14.3 & 15.7 & 16.1 \\
        \hline
        \vspace*{0.1cm}
    \end{tabular}
    \caption{Mean runtime in seconds of the \fkmeans algorithm as a function of $r$ on the $\log \Delta$-testbed dataset described in Section
    \ref{sssec:datasets}. The mean is taken over 5 runs.  Since $\log \Delta$ grows linearly with $r$, this evidences that \fkmeans indeed has a linear
    time-dependency on $\log \Delta$.}
    \label{tbl:logdelta}
\end{table}

\section{Reducing the Impact of the Spread}
\newcommand{\boxsize}{\textsc{MaxDist}}

It is well known that, for $k$-median, embedding the input into a tree using the quadtree decomposition loses only an $O(d \log \Delta)$ factor but that
building this embedding takes time $O(nd \log \Delta)$. However, we will show that we only need to build a fraction of it, allowing for a running time $O(nd
\log \log \Delta)$. 

We preface the discussion by giving an overview of the quadtree embedding algorithm. First, obtain a box enclosing all input points, centered at zero, with all
side length equal to $\boxsize$.\footnote{This can be done as follows: select an arbitrary input point, and translate the dataset so that this point is at the
origin. Then, compute the maximum distance from any point to the origin in time $O(nd)$, and let $\boxsize$ be that distance.} Then, a random shift $s \leq
\boxsize$ is added to all coordinates of all points: the input is now in the box $[-2\boxsize, 2\boxsize]^d$ and this transformation did not change any
distances, henceforth the $k$-median cost is preserved.  The $i$-th level of the tree (for $i \in \lbra 0, ..., \log \Delta \rbra$) is constructed as follows:
a grid of side length $2^{-i} \cdot 2\boxsize$ centered at $0$ is placed, and each cell is a node in the tree.  The parent of a cell $c$ is simply the cell at
level $i-1$ that contains $c$, and the distance between $c$ and its parent is set to $\sqrt{d}2^{-i} \cdot 2\boxsize$ (namely, the $\ell_2$ diameter of $c$).
 
This tree has $O(\log \Delta)$ levels, and it is known that for any solution $\calS$, its cost in the tree metric is within a factor $O(\sqrt d \log \Delta)$ of
its cost for the original metric. Therefore, it is enough for us to find an approximation in the tree metric. We let $\opt_T$ be the optimal solution in the
tree metric, bounded as follows:
 
To remove the $\log\Delta$ dependency, we proceed in two steps: first, we compute a very crude upper-bound on the cost $U$ of the optimal
solution -- up to a $\poly(n)$ factor.  If $U$ is a $c$-approximation of the optimal cost, the natural attempt to reduce the aspect-ratio is to round all
coordinates to multiples of $g = U/(cn)$, giving us a minimum distance of $g$; it is then enough to reduce the diameter to $\poly(n) U$. We proceed as
follows: we place a grid with cell length $O(n \cdot U)$, so that two points from the same cluster in $\opt$ fall into the same cell w.h.p. That way, the
distinct cells do not interact with each other in any reasonable solution.  Then, we compress the input by "moving" non-empty cells closer to each other.

\subsection{Computing a crude upper-bound}

As described, we start by computing an approximate solution $U$ such that $U \leq \poly(n) \cdot \cost(\opt)$ on an.  We focus here on the simpler $k$-median
problem, and show in \cref{app:redKM} how to reduce $k$-means to this case. Our first step is embedding the input into a quadtree as described above. In this
tree, our first lemma shows the necessary approximation exists at the first level for which the input lies in $k+1$ disjoint subtrees. The algorithm to find
this is then a simple binary search through the levels of the tree.

\begin{lemma}\label{lem:apxTree}
Let $i$ be the first level of the decomposition such that at least $k+1$ cells at level $i$ contains any point. Then, $\sqrt{d}2^{-i+1} \cdot \boxsize \leq
\opt_T \leq n \cdot \sqrt{d}2^{-i+4} \cdot \boxsize$.
\end{lemma}

We prove this in Section \ref{app:apx-tree-proof} of the appendix. A direct consequence of this lemma is that the first level of the tree for which at least
$k+1$ cells are non empty provides an $O(n)$-approximation. To count the number of non-empty cells at a given level $i$, one can merely iterate over all
points, for each point identify the cell that contains it (using modulo operation), and store all those cells into a hash table to count the number of elements.
This is done in time $O(nd)$.  Using a binary search on the $O(\log \Delta)$ many levels then concludes this section with the following result:

\begin{lemma}\label{lem:crudeApx}
There is an algorithm running in time $O(nd \log \log \Delta)$ that computes an  $O(n^2 d \log \Delta)$-approximation to $k$-median or $k$-means.
\end{lemma}

\subsection{From Approximate Solution to Small Aspect-Ratio}
Let $U$ be an upper-bound on the optimal cost that is a $c$-approximation, computed via \cref{lem:crudeApx}. We place a grid with side length $d n^2\cdot U$.
The following folklore lemma ensures that with high probability, no cluster of the optimal solution is spread on several grid cells.

\begin{lemma}
The probability that two points $p$ and $q$ are in different grid cells is $O\lpar \frac{\|p-q\|^2}{n^2 U}\rpar$
\end{lemma}

Since $U$ is larger than the distance between any input point and its center in the optimal solution, a union-bound ensures that with probability $1-1/n$, no
cluster of this solution is split among different cells.  In particular, there are at most $k$-non empty cells. We call those "boxes".

From this input, we build a new set of points $P'$ as follows: first, identify the non empty cell (using a hash table as previously). We associate each box with
its center.  For each coordinate $i \in \lbra 1, ..., d \rbra$, sort (in time $O(k \log k)$) the centers according to their value on coordinate $i$. Then, for
each $j \in \lbra 1, ..., k\rbra$, let $c^i_j$ and $c^i_{j+1}$ be the $i$-th coordinate of centers of the $j$-th and $(j+1)$-th boxes. If $c^i_{j+1} - c^i_j
\geq 2d n^2\cdot U$, then for all cells $j'$ with $j' > j$, shift the points of $j'$ by $c^i_{j+1} - c^i_j - 2d n^2\cdot U$ in the $i$-th coordinate.

This can be implemented with a linear scan, and has two effects: first, the diameter of the input is now $\sqrt{d} \cdot 2d n^2\cdot U \cdot k$, as along any
coordinate the maximal distance is $2d n^2\cdot U \cdot k$. Second, two boxes that were adjacent are still adjacent and two boxes that were non-adjacent are
still non-adjacent.

The first property allows us to reduce the aspect-ratio to $(nd \log \Delta)^{O(1)}$.  Indeed, one can round all coordinates to the closest multiple of
$g = \frac{U}{n^4 d^{2} \log \Delta}$. Since every point has moved by at most $g$ and, using \cref{lem:crudeApx}, $U
\leq n^2 d \log(\Delta) \opt$, it is clear that the distance between any point and its rounding is at most $\frac{\opt}{n^2}$. Summing this error over all points,
any solution computed on the gridded data has cost within an additive factor $\pm \frac{\opt}{n}$ of the true cost. Furthermore, the smallest
non-zero distance is $g = \frac{U}{n^4 d^{2} \log \Delta}$, implying that the aspect-ratio of the new metric is $(nd \log \Delta)^{O(1)}$,
as claimed.

The second property, on the other hand, ensures that we can transform a solution $\calS'$ for $P'$ to a solution with exactly the same cost for $P$: in any
(reasonable) solution, points from two non-adjacent boxes will not be in the same cluster in either $P'$ or $P$. Therefore, simply adding back the corresponding
shift to centers of $\calS'$ allows us to transform it to a solution $\calS$. We note that the distance between any point and its closest center does not
change. This is formalized in the next lemma.

\begin{lemma}
Let $\calS'$ be a $c'$-approximation for  $k$-median (resp. $k$-means) on $P'$, where $c' \leq nc$ and $c$ is the approximation guarantee from \cref{lem:crudeApx}. Then, one can compute a solution for $P$ for $k$-median (resp. $k$-means) on $P$, with same cost as $\calS'$ for $P'$, in time $O(nd)$.
\end{lemma}
\begin{proof}

First, since distances in $P'$ are smaller than in $P$, the optimal solution for $P'$ has cost at most $U$. Therefore, two points that are in non-adjacent boxes
(i.e., at distance more than $d n^2\cdot U$) are not in the same cluster of $\calS'$ -- as otherwise $\calS'$ would not be a $c'$-approximation.  Let $\calS$ be
the solution obtained from $\calS'$ by reversing the construction of $P'$. Since this construction preserves adjacency, for all clusters of the solution, all
distances are the same in $P$ and $P'$. Therefore, the costs are equal.

\end{proof}

\subsection*{Extensions.} 
We conclude this section with a few remarks that allow us to generalize \cref{alg:main}.

Consider that \cref{alg:main}  only needs to be provided with an assignment to a solution that is a $O(\polylog k)$ approximationone, implying that one could
compute the solution $\calC$ via any algorithm that satisfies this.  This initial solution may as well be an $O\lpar\polylog \lnor P \rnor_0 \rpar$
approximation: using the iterative coreset construction from \cite{BravermanJKW21}, one could then derive a near-optimal coreset size, only suffering
a $O(\log^* n)$ loss in the running time.

As an example, we illustrate a different approach for $k$-median. One could first embed the input into a hierarchically separated tree (HST) with expected
distortion $O(\log \lnor P \rnor_0)$ \cite{FakcharoenpholRT03}. On such tree metrics, solving $k$-median can be done in linear time using dedicated algorithms
(see e.g. \cite{Cohen-AddadLNSS21}). Using the solution from the HST metric, one can compute a coreset, and iterate using the previous argument.  This embedding
into HST is very similar to what is done by the \fkmeans algorithm, but can be actually performed in \emph{any} metric space, not only Euclidean.  For instance,
in a metric described by a graph with $m$ edges, the running time of this construction would be near linear-time $\tilde O(m)$.




% Acknowledgements should only appear in the accepted version.


\bibliography{references}
\bibliographystyle{icml2022}


%%%%%%%%%%%%%%%%%%%%%%%%%%%%%%%%%%%%%%%%%%%%%%%%%%%%%%%%%%%%%%%%%%%%%%%%%%%%%%%
%%%%%%%%%%%%%%%%%%%%%%%%%%%%%%%%%%%%%%%%%%%%%%%%%%%%%%%%%%%%%%%%%%%%%%%%%%%%%%%
% APPENDIX
%%%%%%%%%%%%%%%%%%%%%%%%%%%%%%%%%%%%%%%%%%%%%%%%%%%%%%%%%%%%%%%%%%%%%%%%%%%%%%%
%%%%%%%%%%%%%%%%%%%%%%%%%%%%%%%%%%%%%%%%%%%%%%%%%%%%%%%%%%%%%%%%%%%%%%%%%%%%%%%
\newpage
\appendix
\onecolumn
\section{Appendix.}



\end{document}