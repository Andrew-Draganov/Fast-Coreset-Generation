
\subsection{Related Work.}

The coreset paradigm has attracted a lot of attention, with a long line of work trying to get the smallest coreset possible in many different metric spaces. The
most prominent example is for Euclidean space \cite{BadoiuHI02, HaM04, Chen09, HuangV20, stoc22}.  In this case, the \textit{group sampling} algorithm developed
in \cite{stoc21, stoc22} yields a coreset of size $\tilde{O}(k\cdot \varepsilon^{-2} \min(k^{z/(z+2)},\varepsilon^{-z}))$ \cite{CLSSS22}, while we know that any
coreset must have size $\Omega \lpar k\eps^{-2}\rpar$ \cite{stoc22}.  Although group sampling has theoretically better bounds than sensitivity sampling, the
experiments of \cite{chrisESA} showed that the later one is likely to be more efficient in practice.

Small size coresets for $k$-median and $k$-means also exist in doubling metrics \cite{huang2018varepsilon}, discrete metrics \cite{FeldmanL11}, metrics induced
by minor-free graphs \cite{BravermanJKW21} or graphs with bounded treewidth \cite{baker2020coresets}.  In finite metrics, the running time $\Omega(nk)$ is
required to compute any approximation to $k$-median or $k$-means \cite{mettu2004optimal}. Since coresets can be used to quickly compute an approximation, the
running time also applies to coreset construction. 

All efficient coreset constructions are probabilistic. This comes with a disadvantage of coresets being difficult to verify and compare. For example, it is
co-NP-hard to check whether a candidate compression is a weak coreset \cite{chrisESA} \footnote{A weak coreset guarantee only requires that a $(1+\varepsilon)$
approximation computed on the coreset yields a $(1+\varepsilon)$ on the entire point set.}. Therefore, although the algorithm succeed with some high
probability, it is unknown how to determine whether the algorithm succeeded or not.  This posed a considerable difficulty for previous experimental evaluations,
where researchers would typically focus on the cost of a solution computed on the designated coreset instead.
%A recent work proposed a pipeline for doing so, and showed that sensitivity sampling algorithm slightly outperforms the theoretically best group sampling
%\cite{chrisESA}.  We will use this pipeline to evaluate the quality of the coresets analyzed in this paper.  We will use a similar heuristic to compo

