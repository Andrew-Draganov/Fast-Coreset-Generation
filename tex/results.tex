\section{Fast Compression in Practice}

Despite the near-linear time algorithm described in Sections~\ref{sec:theory} and~\ref{sec:logdelta}, the coreset construction in \ref{alg:main} nonetheless
requires an $O(1)$ approximation to the original task before the sampling can occur. Although theoretically justified, it is unclear how necessary this is in
practice -- would a worse approximation factor still be representative of the dataset for practical purposes? We answer this question by defining a suite of
algorithms, datasets and evaluation procedures that allow for a comprehensive study of speed vs. accuracy tradeoffs in compression for clustering.  We begin by
describing our experimental setup before showing results in both the static and streaming settings.  We note that these results correspond directly to
downstream clustering tasks.

\subsection{Experimental Setup}
\subsubsection{Metrics}
\label{sssec:metrics}

We analyze the sampling methods along two metrics -- compression accuracy and construction time. Although measuring runtime is standard, it is unclear how to
confirm that a subset of points satisfies the coreset property over all solutions. To this end, we use the distortion measure introduced in~\cite{chrisESA} $
\max \left( \dfrac{\cost(P, \calC_{\Omega})}{\cost(\Omega, \calC_{\Omega})}, \dfrac{\cost(\Omega, \calC_{\Omega})}{\cost(P, \calC_{\Omega})} \right),$ where
$\calC_{\Omega}$ is a candidate solution that was computed over the coreset $\Omega$. This will be within $1+\varepsilon$ if the coreset guarantee is satisfied
but may be unbounded otherwise.  We refer to this metric as the \emph{coreset distortion}.

\subsubsection{Algorithms}
\label{ssec:algorithms}

We compare Fast-Coresets (Algorithm~\ref{alg:main}) against 4 different benchmark sampling strategies that span the space between optimal time and optimal
accuracy.
\begin{itemize}
        \item \emph{Standard uniform sampling}. Each point is sampled with equal probability and weights are set to $n / m$, where $m$ is the size of the sample.
        \item \emph{Standard sensitivity sampling \cite{LS10}}. We refer to Section~\ref{ssec:sens_sampling}.
        \item \emph{Lightweight coresets \cite{bachem2018scalable}}. Obtain a coreset by sampling sensitivity values with respect to the $1$-means solution,
            i.e. sensitivities are given by $\hat{s}(p) = 1/|P| + \cost(p, \mu) / \cost(P, \mu)$, where $\mu$ is the dataset mean.
        \item \emph{Welterweight coresets}. For any $j \in \{1,..., k\}$, we compute a coreset using sensitivity sampling with respect to a candidate
            $j$-means++ solution.
\end{itemize}

We take a moment here to motivate the welterweight coreset algorithm.  Consider that lightweight coresets are solving $1$-means to obtain sensitivity
values whereas sensitivity sampling is solving the $k$-means problem. Given that sensitivity values are estimated via $ \sigma_\calC(p) = \dfrac{\cost(p,
\calC)}{\cost(\calC_{p}, \calC)} + \dfrac{1}{|\calC_p|},$ changing the value of $j$ affects the cluster sizes $|\calC_p|$ and therefore acts as a direct
interpolation between uniform and sensitivity sampling.  We default to $j = \log k$ unless stated otherwise. We will use the term `accelerated sampling methods'
when referring to uniform, lightweight and welterweight coresets as a group.

\subsubsection{Datasets}
\label{sssec:datasets}

We employ several real and artificial datasets to evaluate the quality of a coreset.  For our real-world data, we utilize the Adult~\cite{Dua:2019},
MNIST~\cite{mnist}, Song~\cite{song}, Census~\cite{census}, and Cover Type~\cite{covtype} datasets, whose characteristics are summarized in
Table~\ref{tbl:datasets} in the Appendix. These are standard datasets for coreset and clustering evaluations.
We complement these with artificial datasets and default to $n = 50\,000$ and $d=50$ unless stated otherwise.
\begin{itemize}

    \item \emph{c-outlier}. Place $n-c$ points in a single location and $c$ points a large distance away.

    \item \emph{Geometric}. Place $c k$ points at $(1, 0, 0, \cdots)$, $\frac{ck}{r}$ points at $(0, 1, 0, \cdots)$, $\frac{ck}{r^2}$ points
        at $(0, 0, 1, \cdots)$, and so on for $\log_r (ck)$ rounds. Thus, the data creates a high-dimensional simplex with uneven weights across the vertices. We
        default to $c = 100$ and $r=2$.

    \item \emph{Gaussian mixture}. A set of scattered Gaussian clusters of varying density. These clusters are sequentially defined, with the size of the first
        cluster defined by $\frac{n}{\kappa} \exp \left( \gamma \cdot \rho_0 \right)$, where $\kappa$ is the number of Gaussian clusters, $\rho_0$ is uniformly
        chosen from $[-0.5, 0.5]$, and $\gamma$ is a hyperparameter that affects the distribution of cluster sizes.  Then, given clusters $\{c_1, \cdots,
        c_i\}$, we obtain the size of the $(i+1)$-st cluster by $|c_{i+1}| = \frac{n - \sum_i |c_i|}{\kappa - i}\exp \left( \gamma \cdot \rho_{i+1} \right).$

        This has the property that all clusters have size $n / k$ when $\gamma = 0$ and, as $\gamma$ grows, the cluster sizes diverge at an exponential rate.
        We note that this is a well-clusterable instance with respect to cost stability conditions, see \cite{AwS12,Cohen-AddadS17,KuK10,ORSS12}.

    \item \emph{Benchmark}. A specific distribution of points introduced in \cite{chrisESA} as a testbed for coreset algorithms.  It has the property that
        all reasonable $k$-means solutions are of equal quality but are maximally far apart in the solution space. Thus, the dataset is fully determined by the
        number of centers $k$. We follow the advice in the benchmark's original presentation and produce three benchmark datasets of varying size before
        applying random offsets to each. We choose the sizes by $k_1 = \frac{k}{c_1}$, $k_2 = \frac{k - k_1}{c_2}$, and $k_3 = k - k_1 - k_2$ for $c_1, c_2 \in
        \mathbb{R}^+$.

\end{itemize}

The artificial datasets are constructed to emphasize strengths and weaknesses of the various sampling schemas. For example, the $c$-outlier problem contains
very little information and, as such, should be simple for any sampling strategy that builds a reasonable representation of its input. The geometric dataset
then increases the difficulty by having $\log k$ clusters and, subsequently, more regions of interest that must be sampled. The Gaussian mixture dataset is
harder still, as it incorporates uneven inter-cluster distances and inconsistent cluster sizes. Lastly, the benchmark dataset is devised to be a worst-case
example for sensitivity sampling. We discuss the parameters and implementation specifics of the datasets in~\cref{app:data_params}. We run our experiments on
an Intel Core i9 10940X 3.3GHz 14-Core processor without parallelization.

\subsection{Evaluating Sampling Strategies}
\label{ssec:alg_qualities}

\begin{table*}
    \centering
    \vspace*{-0.1cm}
    \tiny
    \begin{tabular}{|c|cc|cc|cc|cc|}
        \hline
        & \multicolumn{8}{c|}{Method} \\
        \cline{2-9} & \multicolumn{2}{c|}{Uniform Sampling} & \multicolumn{2}{c|}{Lightweight} & \multicolumn{2}{c|}{Welterweight} & \multicolumn{2}{c|}{Fast Coreset} \\
        & $m=40k$ & $m=80k$ & $m=40k$ & $m=80k$ & $m=40k$ & $m=80k$ & $m=40k$ & $m=80k$ \\
        \cline{2-9}
        $c$-outlier & 405 $\pm$ 24K & 159 $\pm$ 20K & 1.07 $\pm$ 0.0 & 5.63 $\pm$ 84.6 & 9.44 $\pm$ 105 & 1.03 $\pm$ 0.0 & 1.12 $\pm$ 0.0 & 1.05 $\pm$ 0.0 \\
        Geometric & 86.3 $\pm$ 8.4K & 21.8 $\pm$ 652 & 1.05 $\pm$ 0.0 & 11.6 $\pm$ 452 & 1.11 $\pm$ 0.0 & 1.03 $\pm$ 0.0 & 1.11 $\pm$ 0.0 & 1.05 $\pm$ 0.0 \\
        Gaussian Mix. & 3.17 $\pm$ 3.42 & 1.43 $\pm$ 0.25 & 2.64 $\pm$ 1.61 & 1.81 $\pm$ 0.58 & 2.26 $\pm$ 1.55 & 1.28 $\pm$ 0.07 & 1.24 $\pm$ 0.0 & 1.13 $\pm$
        0.0 \\
        \makecell{Benchmark\\---} & \makecell{1.07 $\pm$ 0.0\\---} & \makecell{1.03 $\pm$ 0.0\\---} & \makecell{1.11 $\pm$ 0.0\\---} & \makecell{1.05 $\pm$
        0.0\\---} & \makecell{1.10 $\pm$ 0.0\\---} & \makecell{1.04 $\pm$ 0.0\\---} & \makecell{1.15 $\pm$ 0.0\\---} & \makecell{1.06 $\pm$ 0.0\\---} \\
        MNIST & 1.08 $\pm$ 0.0 & 1.03 $\pm$ 0.0 & 1.08 $\pm$ 0.0 & 1.03 $\pm$ 0.0 & 1.08 $\pm$ 0.0 & 1.04 $\pm$ 0.0 & 1.08 $\pm$ 0.0 & 1.04 $\pm$ 0.0 \\
        Adult & 1.08 $\pm$ 0.0 & 1.04 $\pm$ 0.0 & 1.09 $\pm$ 0.0 & 1.04 $\pm$ 0.0 & 1.32 $\pm$ 0.0 & 1.17 $\pm$ 0.0 & 1.17 $\pm$ 0.0 & 1.07 $\pm$ 0.0 \\
        Song & 1.30 $\pm$ 0.0 & 1.14 $\pm$ 0.0 & 1.13 $\pm$ 0.0 & 1.12 $\pm$ 0.0 & 1.18 $\pm$ 0.0 & 1.16 $\pm$ 0.0 & 1.50 $\pm$ 0.0 & 1.29 $\pm$ 0.0 \\
        Census & 1.15 $\pm$ 0.0 & 1.07 $\pm$ 0.0 & 1.12 $\pm$ 0.0 & 1.06 $\pm$ 0.0 & 1.14 $\pm$ 0.0 & 1.06 $\pm$ 0.0 & 1.13 $\pm$ 0.0 & 1.07 $\pm$ 0.0 \\
        Cover-Type & 1.12 $\pm$ 0.0 & 1.05 $\pm$ 0.0 & 1.11 $\pm$ 0.0 & 1.06 $\pm$ 0.0 & 1.17 $\pm$ 0.0 & 1.08 $\pm$ 0.0 & 1.11 $\pm$ 0.0 & 1.04 $\pm$ 0.0 \\
        \hline
    \end{tabular}
    \caption{Distortion means and variances for different sample sizes across datasets; taken over 5 runs.}
    \label{tbl:distortion}
    \vspace*{0.1cm}
    \tiny
    \begin{tabular}{|c|cc|cc|cc|cc|}
        \hline
        & \multicolumn{8}{c|}{Method} \\
        \cline{2-9} & \multicolumn{2}{c|}{Uniform Sampling} & \multicolumn{2}{c|}{Lightweight} & \multicolumn{2}{c|}{Welterweight} & \multicolumn{2}{c|}{Fast Coreset} \\
        & Streaming & Static & Streaming & Static & Streaming & Static & Streaming & Static \\
        \cline{2-9}
        $c$-outlier & 221 $\pm$ 15K & 261 $\pm$ 44K & 1.07 $\pm$ 0.0 & 5.51 $\pm$ 78.9 & 1.09 $\pm$ 0.0 & 12.1 $\pm$ 80.1 & 1.13 $\pm$ 0.0 & 1.11 $\pm$
        0.0 \\
        Geometric & 66.5 $\pm$ 2.7K & 140 $\pm$ 1.8K & 85.2 $\pm$ 2.8K & 45.6 $\pm$ 4.2K & 1.09 $\pm$ 0.0 & 10.4 $\pm$ 349 & 1.15 $\pm$ 0.0 & 1.12 $\pm$ 0.0 \\
        Gaussian Mix. & 1.51 $\pm$ 0.07 & 2.42 $\pm$ 2.52 & 2.35 $\pm$ 0.67 & 2.38 $\pm$ 1.78 & 1.45 $\pm$ 0.05 & 3.65 $\pm$ 3.85 & 1.15 $\pm$ 0.0 & 1.24 $\pm$
        0.0 \\
        Benchmark & 1.10 $\pm$ 0.0 & 1.07 $\pm$ 0.0 & 1.08 $\pm$ 0.0 & 1.11 $\pm$ 0.0 & 1.09 $\pm$ 0.0 & 1.11 $\pm$ 0.0 & 1.18 $\pm$ 0.0 & 1.16 $\pm$ 0.0 \\
        MNIST & 1.42 $\pm$ 0.0 & 1.08 $\pm$ 0.0 & 1.07 $\pm$ 0.0 & 1.07 $\pm$ 0.0 & 1.02 $\pm$ 0.0 & 1.09 $\pm$ 0.0 & 1.12 $\pm$ 0.0 & 1.08 $\pm$ 0.0 \\
        Adult & 1.33 $\pm$ 0.0 & 895K $\pm$ 3.2B & 1.09 $\pm$ 0.0 & 1.09 $\pm$ 0.0 & 1.27 $\pm$ 0.01 & 1.32 $\pm$ 0.0 & 1.14 $\pm$ 0.0 & 1.15 $\pm$ 0.0 \\
        \hline
    \end{tabular}
    \caption{Distortion means and variances in the streaming and non-streaming setting for each method.}
    \label{tbl:composition}
\end{table*}


\begin{table}[htbp]
    \centering
    \small
    \begin{tabular}{lcccc}
        Algorithm & $\gamma = 0$ & $\gamma = 1$ & $\gamma = 3$ & $\gamma = 5$\\
        \hline
        \vspace*{-0.15cm}\\
        LW Coreset & 1.03 & 1.03 & 1.36 & 2.17\\
        $j=2$ & 1.04 & 1.04 & 1.04 & 1.92\\
        $j=\log k$ & 1.04 & 1.04 & 1.04 & 1.95\\
        $j=\sqrt{k}$ & 1.05 & 1.06 & 1.04 & 1.18\\
        Fast Coreset & 1.03 & 1.03 & 1.04 & 1.12\\
        \vspace*{-0.15cm}\\
    \end{tabular}
    \caption{The effect of $\gamma$ in the Gaussian mixture dataset on the coreset distortion. We report the means over 5 random dataset generations.
    Each generation had 50\,000 points in 50 dimensions, with 50 Gaussian clusters and coresets of size 4\,000. We set $k=100$.}
    \label{tbl:class-imbalance}
\end{table}



\paragraph*{Theoretically guaranteed methods.}

We first compare the Fast-Coreset algorithm with standard sensitivity sampling to ensure that it obtains appropriate accuracy in a shorter runtime.  To this
end, the last columns of Tables~\ref{tbl:composition} and~\ref{tbl:distortion} show that the Fast-Coreset method produces compressions of consistently low
distortion and that this holds across datasets, $m$-scalar values and in the streaming setting.  Despite this, Figure~\ref{fig:coreset_size_on_sens_quality}
shows that varying $k$ from $50$ to $400$ causes a linear slowdown in sensitivity sampling but only a logarithmic one for the Fast-Coreset method. This analysis
confirms the theory in Section~\ref{sec:theory} and we therefore do not add traditional sensitivity sampling to the other experiments.

\paragraph*{Speed vs. Accuracy.}

We now refer the reader to the remaining columns of Table~\ref{tbl:distortion}, where we show the effect of coreset size across the real-world datasets by
sweeping over $m$-scalar values. Despite the suboptimal theoretical guarantees of the accelerated sampling methods, we see that they obtain competitive
distortions to sensitivity sampling while also running faster than Fast-Coresets in practice. We attribute this to the well-behaved nature of the real-world
datasets, where they have few outliers and classes of consistent sizes.

To verify this, consider the results of these sampling strategies on the artificial datasets in Table~\ref{tbl:distortion} -- as disparity in cluster sizes and
distributions grows, the accelerated sampling methods have difficulty capturing all of the outlying points in the dataset.  While uniform sampling is expected
to be brittle, it may be less obvious what causes light- and welterweight coresets to break. Consider that the lightweight coreset is biased towards points that
are far from the mean and, as a simple counterexample, is likely to miss a small cluster that is close to $\mu$.  Now let $\calC_j$ be the approximation
obtained during welterweight coresets and observe that the sum of importance values of the points belonging to center $c_i \in \calC_j$ is
$\sum_{p \in c_i} \left[ \dfrac{\cost(p, c_i)}{\cost(c_i, \calC_j)} + \dfrac{1}{|c_i|} \right] = \dfrac{\sum_{p \in c_i} \cost(p,
    c_i)}{\cost(c_i, \calC_j)} + 1 = 2$

Thus, our probability mass is distributed across the clusters that have been found in the approximate solution. Naturally, if $j < k$ and we missed a cluster
from $\opt$, there is some set of points that have not received an appropriate probability mass and may therefore be missed.

We evaluate the full extent of this relationship in Table~\ref{tbl:class-imbalance}, where we show the interplay between the welterweight coreset's $j$
parameter and the Gaussian mixture dataset's $\gamma$ parameter. We can consider this as answering the question ``How good must our approximate solution be
before sensitivity sampling can handle class imbalance?'' To this end, all the methods have low distortion for small values of $\gamma$ but, as $\gamma$ grows,
only Fast-Coresets\footnote{And, to a lesser extent, welterweight coresets for larger values of $j$.} are guaranteed to have low distortion.

This pattern changes in the benchmark dataset as it is devised so that maximally different $k$-means solutions give similar quality. Thus, it challenges
sensitivity sampling's reliance on the initial solution but we see that, despite this, sensitivity sampling does not require significantly larger sample sizes
before it provides good compression. We lastly show how these compressions facilitate fast clustering on large datasets in Table~\ref{tbl:lloyds}. We see that
running Lloyd's algorithm on the samples obtains a consistent cost value on the full datasets.

%\begin{table*}
    \centering
    \small
    \begin{tabular}{|c|cccc|}
        \hline
        & Uniform Sampling & Lightweight & Welterweight & Fast Coreset \\
        MNIST & 12450 & 12379 & 12512 & 12449 \\
        Adult & 24982 & 24935 & 24944 & 25013 \\
        Song & 16753 & 16612 & 16950 & 17010 \\
        Census & 36697 & 35577 & 37379 & 32163 \\
        Cover-Type & 10506 & 10337 & 10032 & 9787 \\
        \hline
    \end{tabular}
    \caption{$cost(P, \calC_S)$, where $P$ is the whole dataset and $\calC_S$ is found via Lloyd's algorithm on the coreset. Sample sizes down the rows are
    $m=4K,4K,20K,20K,20K$ and initializations are identical within each row. We only show the first 5 digits of the cost for readability.}
    %, i.e. uniform sampling on MNIST has error 12\,450\,330\,577
    \label{tbl:lloyds}
\end{table*}


%\begin{table*}
    \centering
    \tiny
    \begin{tabular}{|c|cc|cc|cc|cc|}
        \hline
        & \multicolumn{8}{c|}{Method} \\
        \cline{2-9} & \multicolumn{2}{c|}{Uniform Sampling} & \multicolumn{2}{c|}{Lightweight} & \multicolumn{2}{c|}{Welterweight} & \multicolumn{2}{c|}{Fast Coreset} \\
        & Streaming & Static & Streaming & Static & Streaming & Static & Streaming & Static \\
        \cline{2-9}
        $c$-outlier & 221 $\pm$ 15K & 261 $\pm$ 44K & 1.07 $\pm$ 0.0 & 5.51 $\pm$ 78.9 & 1.09 $\pm$ 0.0 & 12.1 $\pm$ 80.1 & 1.13 $\pm$ 0.0 & 1.11 $\pm$
        0.0 \\
        Geometric & 66.5 $\pm$ 2.7K & 140 $\pm$ 1.8K & 85.2 $\pm$ 2.8K & 45.6 $\pm$ 4.2K & 1.09 $\pm$ 0.0 & 10.4 $\pm$ 349 & 1.15 $\pm$ 0.0 & 1.12 $\pm$ 0.0 \\
        Gaussian Mix. & 1.51 $\pm$ 0.07 & 2.42 $\pm$ 2.52 & 2.35 $\pm$ 0.67 & 2.38 $\pm$ 1.78 & 1.45 $\pm$ 0.05 & 3.65 $\pm$ 3.85 & 1.15 $\pm$ 0.0 & 1.24 $\pm$
        0.0 \\
        Benchmark & 1.10 $\pm$ 0.0 & 1.07 $\pm$ 0.0 & 1.08 $\pm$ 0.0 & 1.11 $\pm$ 0.0 & 1.09 $\pm$ 0.0 & 1.11 $\pm$ 0.0 & 1.18 $\pm$ 0.0 & 1.16 $\pm$ 0.0 \\
        MNIST & 1.42 $\pm$ 0.0 & 1.08 $\pm$ 0.0 & 1.07 $\pm$ 0.0 & 1.07 $\pm$ 0.0 & 1.02 $\pm$ 0.0 & 1.09 $\pm$ 0.0 & 1.12 $\pm$ 0.0 & 1.08 $\pm$ 0.0 \\
        Adult & 1.33 $\pm$ 0.0 & 895K $\pm$ 3.2B & 1.09 $\pm$ 0.0 & 1.09 $\pm$ 0.0 & 1.27 $\pm$ 0.01 & 1.32 $\pm$ 0.0 & 1.14 $\pm$ 0.0 & 1.15 $\pm$ 0.0 \\
        \hline
    \end{tabular}
    \caption{Distortion means and variances in the streaming and non-streaming setting for each method. We default to $m=20k$.}
    \label{tbl:composition}
\end{table*}



\subsection{Streaming Setting}
\label{ssec:streaming}

One of the most common use-cases for big-data algorithms is the streaming setting, where one receives input in batches and must maintain a compression that is
representative of the dataset. Although there is a wealth of sampling and coreset methods in the streaming paradigm, we require consistency
across algorithms and therefore assume a black-box sampling procedure. Since the coreset property is preserved under composition, we utilize the merge-\&-reduce
strategy originally proposed by \cite{BS80} and first applied to maintaining clustering coresets in stream by \cite{HaM01}. The idea is to first partition the
input into $b$ blocks and then perform sampling and composition along them until a single compression is obtained. Specifically, we start by obtaining
a coreset\footnote{Although we use the word coreset here, this process can also be applied to the other sampling procedures.} on each block. Then, combining
samples using a complete binary tree, we (1) recursively re-sample from the children until there is at least one coreset for each level\footnote{If there are
$b=8$ blocks, then there would be coresets corresponding to blocks $[[1], [2], [3, 4], [5, 6, 7, 8]]$} in the tree and then (2) concatenate these samples and
obtain one final coreset from the composition. Since we are composing coresets from coresets, the errors stack and, in theory, we should require more samples
to obtain a similar accuracy.

Despite this, we see the opposite result for many of our sampling strategies. Surprisingly, the accelerated sampling methods all perform \emph{better} under
composition on the artificial datasets and do not suffer significant drops in accuracy or variance on the real datasets\footnote{We show the effect of
composition on runtime in \ref{fig:streaming_runtimes} in the Appendix.}. Although inconsistent with the prevailing intuition, we must therefore conclude that
the accelerated sampling methods are \emph{more} feasible in the streaming setting but that they still suffer from the lack of a full coreset guarantee.  We
suspect that this is due to the non-uniformity imposed by the merge-\&-reduce algorithm. To see this, consider uniform sampling on the $c$-outlier dataset
during the final step of the composition, where we are composing the samples corresponding to each layer of the tree. Assume first that our outlier points
happened to fall in the first block. Then we have taken a sample of size $m$ from this block and immediately use this for the final composition. Thus, in this
case the outliers are \emph{more} likely to be in our final sample than in the non-streaming setting. In the alternate setting where the outliers are less
likely, consider that our expected error is already high, so missing the outlier `more' does not significantly affect our expected error. 
